\documentclass[UTF-8]{ctexbeamer}
\usetheme{Berkeley}

\usepackage{multimedia}
\usepackage{listings}
\usepackage{minted}

\title{How NOT to Rust}

\author{喵喵}
\date{2021.10}

\begin{document}
\begin{frame}
  \titlepage
  \begin{center}
    \includegraphics[width=.1\textwidth]{assets/float.png}
  \end{center}
\end{frame}

\section{Intro}

\begin{frame}
  \frametitle{How NOT to Rust}

  本来想用的标题是 \texttt{A Brief, Incomplete, and Mostly Wrong History of Rust}\footnote{\url{http://james-iry.blogspot.com/2009/05/brief-incomplete-and-mostly-wrong.html}}。

  \pause
  \vspace{1em}

  但是喵喵写 Slide 毫无灵感\dots

  \begin{figure}
    \includegraphics<2>[width=0.25\textwidth]{assets/avatar.jpg}
    \includegraphics<3>[width=0.5\textwidth]{assets/avatarWD40.png}
    \only<2>{\caption{呼呼喵喵}}
    \only<3>{\caption{精神喵喵!!}}
  \end{figure}
\end{frame}

\begin{frame}
  \frametitle{How NOT to Rust}

  来聊聊 Rust 的历史。

  \pause

  这个 Talk 包括:
  \begin{itemize}
    \item Rust 简易入门!
    \item Rust 的各种 Joke
  \end{itemize}

  这个 Talk 不包括:
  \begin{itemize}
    \item Rust 详细入门,请查看 The Rust Programming Language
    \item Rust 详细语义,请查看 The Rust Language Reference
    \item 讲者存在 Rust 深刻知识的任何可能。
  \end{itemize}

  \pause

  喵喵刚刚入门 Rust,请爱护喵喵!
\end{frame}

\section{The Uglies}

\begin{frame}
  \frametitle{快速 Rust 入门}

  Rust 和 \texttt{C(|++), Java(|Script), Python, LISP, \dots} 有什么不同?

  \pause
  \vspace{1em}

  \begin{itemize}
    \item Algebraic Data Types
    \item \texttt{crate} and \texttt{module}
    \item Polymorphism: Generic + Trait
    \item (The dreaded) Lifetime / \texttt{borrowck}
  \end{itemize}
\end{frame}

\begin{frame}[fragile]
  \frametitle{ADT: Algebraic Data Types}

  \begin{minted}{rust}
// KV DB, Client -> Server conn payload
enum KVPayload {
  Close,
  Get(String),
  Put {
    key: String,
    value: String,
    expire: Datetime,
  },
}
  \end{minted}

  \pause
  \vspace{1em}
  Disjoint sum over products!
\end{frame}

\begin{frame}[fragile]
  \frametitle{Module-level encapsulation}

  Crate = Node/Go/Python package (sort of)

  功能集合,例如:\texttt{serde} 提供了序列化、反序列化相关的基础设施。

  \pause
  \vspace{1em}

  Module = Java package (sort of)

  实现单元,“可见性“的边界。例如:\texttt{serde::ser} 包含序列化 (Serialize) 相关的声明和实现。

  \pause
  \vspace{1em}

  \begin{minted}{rust}
mod data;
pub use data::{Input, Output};
  \end{minted}
\end{frame}

\begin{frame}[fragile]
  \frametitle{Abusing modules}

  \texttt{library/std/src/os/mod.rs:}

  \begin{minted}[breaklines]{rust}
// unix
#[cfg(not(all(
    doc,
    any(
        all(target_arch = "wasm32", not(target_os = "wasi")),
        all(target_vendor = "fortanix", target_env = "sgx")
    )
)))]
#[cfg(target_os = "hermit")]
#[path = "hermit/mod.rs"]
pub mod unix;
  \end{minted}
\end{frame}

\begin{frame}
  \frametitle{Polymorphism}

  \begin{itemize}
    \item Composition over Inheritance {\tiny{设计模式!}}
    \item Trait: 描述一个接口
    \pause
    \item Generic: 使用 Trait 的静态派发
    \item Trait Objects: 使用 Trait 的动态派发
  \end{itemize}
\end{frame}

\begin{frame}[fragile]
  \frametitle{Polymorphism, Cont.}
  \begin{minted}{rust}
trait Monoid {
  // "Member functions"
  fn product(&self, other: &Self) -> Self;
  // "Static functions"
  fn identity() -> Self;
  fn is_commutative() -> bool;
}

trait Group: Monoid {
  fn inverse(&self) -> Self;
  // Default impl
  fn is_abelian() -> bool {
    return Self::is_commutative();
  }
}
  \end{minted}
\end{frame}

\begin{frame}[fragile]
  \frametitle{Polymorphism, Cont.}

  \begin{minted}{rust}
fn subgroup_by<G>(gen: G) -> Vec<G>
  where G: Group + Eq + Clone
{
  let mut cur = gen.clone();
  let mut result = Vec::new();
  loop {
    result.push(cur.clone());
    cur = cur.product(gen);
    if cur == gen {
      return result;
    }
  }
}
  \end{minted}
\end{frame}

\begin{frame}[fragile]
  \frametitle{Polymorphism, Cont.}

  \begin{minted}{rust}
    let trait_obj: &dyn Group = &group;
  \end{minted}

  \pause
  \vspace{1em}

  VTable with fat pointer
\end{frame}

\begin{frame}
  \frametitle{Borrow checker!}

  核心目标:
  \begin{itemize}
    \item 读不了非法内存 (Uninitialized, Use after freed)
    \item 比较难 Race (一段代码、一个线程在读,另外一段代码、一个线程在写)
  \end{itemize}

  \pause
  \vspace{1em}

  Lifetime!
\end{frame}

\begin{frame}
  \frametitle{The uglies}
\end{frame}

\begin{frame}
  \frametitle{That's All!}

  \begin{center}
    \includegraphics[width=.5\textwidth]{assets/look.png}

    Question time!
  \end{center}
\end{frame}
\end{document}
