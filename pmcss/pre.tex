\documentclass[UTF-8]{ctexbeamer}
\usetheme{Berkeley}

\usepackage{multimedia}
\usepackage{listings}
\usepackage{minted}

\title{(后)现代前端 CSS 技术}


\author{喵喵}
\date{2020.10}

\begin{document}
\begin{frame}
  \titlepage
  \begin{center}
    \includegraphics[width=.1\textwidth]{assets/float.png}
  \end{center}
\end{frame}

\begin{frame}
  \frametitle{喵喵的(后)现代 CSS}

  每个人的 CSS 写法都不一样
  
  \vspace{1em}
  
  喵喵喜欢的功能和实践会着重于以下几点特征:

  \begin{itemize}
    \item 实现的性能(60fps)
    \item 代码可维护性
    \item 最好存在 Fallback
  \end{itemize}
\end{frame}

\begin{frame}
  \frametitle{What's in the box}
  \begin{itemize}
    \item CSS Basics
    \item 一些新功能
    \begin{itemize}
      \item CSSOM/Houdini, WAAPI, etc.
    \end{itemize}
    \item 喵喵的 practice
    \begin{itemize}
      \item SASS, BEM, etc.
    \end{itemize}
  \end{itemize}

  \pause

  \vspace{2em}
  例子:
  \begin{itemize}
    \item \url{https://meow.plus}
    \item \url{https://gust.construction}
  \end{itemize}
\end{frame}

\section{CSS Transition}
\begin{frame}[fragile]
  \frametitle{CSS Optimization Basics}
  \begin{minted}{css}
    .jiege + .meow {
      opacity: 0;
      transition: opacity .2s ease;
    }
  \end{minted}
  \pause
  $$\Downarrow$$
  \begin{minted}{html}
    <main>
      <h1 class="jiege">God himself</h1>
      <small class="meow">Meow</small>
    </main>
  \end{minted}
\end{frame}

\begin{frame}
  \frametitle{CSS Optimization Basics (Cont.)\footnote{\url{https://developers.google.com/web/fundamentals/performance/rendering}}}

  \begin{itemize}
    \item 找到 CSS 属性发生变化的元素: \texttt{main > small.meow}
    \pause
    \item 计算 Diff,并且观察是不是有 Transition / Animation
    \pause
    \item 本次重绘 Tick(如果是持续变化,那么在接下来的每一个 Tick):
    \pause
    \begin{itemize}
      \item 计算 Style set
      \item Layout: 重新计算元素的尺寸
      \item Paint: 重新绘制元素
      \item Composite: 将新绘制的元素叠加到图层上
    \end{itemize}
  \end{itemize}

  \pause
  \vspace{2em}

  Layout 和 Paint 可以在不必须的情况下被省去(e.g. \texttt{opacity} 的变化)

  \pause
  \vspace{1em}

  在大量使用 Transition 的场景下,Layout 是非常慢的。Paint 会消耗更多的 GPU 带宽,相比 Composite 也更慢,最好也省去。
\end{frame}

\begin{frame}
  \frametitle{选定过渡属性}

  稍稍过时的数据: \url{https://csstriggers.com/}

  \vspace{2em}

  \begin{itemize}
    \item \texttt{transform}
    \item \texttt{opacity}
    \item \texttt{filter}\footnote{\url{https://www.chromium.org/developers/design-documents/image-filters}}
    \item \texttt{clip-path}\footnote{\url{https://groups.google.com/a/chromium.org/g/paint-dev/c/3bXUo0X3C5I}}
    \item \texttt{backdrop-filter}??
  \end{itemize}

\end{frame}

\begin{frame}
  \frametitle{产生了一些问题...}

  目标效果:一个容器的宽度连续变化。

  \vspace{2em}
  \pause

  \begin{itemize}
    \item 可以用 \texttt{width},但是会很慢。
    \pause
    \item \texttt{scale}? 最开始会有一次突变
    \pause
    \item 在父亲元素上用 \texttt{clip-path},在孩子元素上用 \texttt{translate}
  \end{itemize}

  \pause
  \vspace{2em}
  怎么让“宽度”这个变量变化?
\end{frame}

\begin{frame}[fragile]
  \frametitle{产生了一些问题... (Cont.)}

  朴素的方案:可以每个用到的地方分别写一个 class

  \begin{minted}{css}
    .parent { transition: clip-path .2s ease; }
    .left-child { transition: transform .2s ease; }

    .parent.shrink {
      clip-path: polygon(
        50vw 0, 50vw 100vh, 100vw 100vh, 100vw 0);
    }
    .parent.shrink .left-child {
      transform: translateX(50vw);
    }
  \end{minted}

  \pause

  \vspace{2em}
  如果共享的状态涉及大量元素呢,或者有大量不同的状态呢?
  
  e.g. 暗色模式,多种不同的宽度,etc...
\end{frame}

\section{Custom Property}
\begin{frame}[fragile]
  \frametitle{CSS Variable / Custom Property}

  \begin{minted}{css}
    .parent {
      --slice: 0vw;
      clip-path: polygon(
        var(--slice) 0, var(--slice) 100vh, 
        100vw 100vh, 100vw 0,
      );
      transition: clip-path .2s ease;
    }
    .parent.shirnk { --slice: 50vw; }
    .left-child {
      transition: transform .2s ease;
      transform: translateX(var(--slice));
    }
  \end{minted}
\end{frame}

\begin{frame}[fragile]
  \frametitle{CSS Variable / Custom Property (Cont.)}
  \texttt{--var-name: "Anything here"}

  \begin{itemize}
    \item 可以继承
    \item \textbf{无类型}
    \item 未设置时无初始值,因此在使用的时候相当于非法属性值。
  \end{itemize}

  \pause
  \vspace{2em}
  \begin{minted}{css}
    transition: display 10s ease;
  \end{minted}
  \pause
  \begin{minted}{css}
    transition: --var-name 10s ease;
  \end{minted}
\end{frame}

\begin{frame}[fragile]
  \frametitle{Houdini: @property}
  \begin{minted}{css}

    @property --shift-width {
      syntax: '<length>';
      inherits: true;
      initial-value: 0;
    }

  \end{minted}

  \pause
  \vspace{1em}

  \texttt{--shift-width} 现在是有类型的了,因此以下过渡可以正常工作:

  \vspace{1em}

  \begin{minted}{css}
    transition: --shift-width .2s ease;
  \end{minted}

  \pause
  \vspace{1em}

  同时能保证在同一 tick,所有用到这个变量的 Transition 都是同步的。
\end{frame}

\begin{frame}[fragile]
  \frametitle{Houdini: @property (Cont.)}

  能不能再给力一些?

  \pause
  \vspace{2em}

  \begin{minted}{css}
    .foobar {
      transition:
        --first-stage .2s 0s ease,
        --second-stage .2s .1s ease;

      transform: translateX(calc(
        var(--first-stage) + var(--second-stage)
      ));
    }
  \end{minted}
\end{frame}

\section{JS Interop}

\begin{frame}[fragile]
  \frametitle{WAAPI}

  \begin{block}{Web Animation API}
    [WAAPI] defines a model and an API for interacting with it, for synchronization and timing of changes to the presentation of a Web page.\footnote{\url{https://drafts.csswg.org/web-animations-1/}}
  \end{block}

  \pause

  \begin{minted}{javascript}
    const animation = el.animate(
      [{ transform: 'translateY(0)' }],
      [{ transform: 'translateY(100px)' }],
      {
        duration: 1000, delay: 1000,
        easing: 'ease', fill: 'both'
      },
    );
    await animation.finished;
  \end{minted}
\end{frame}

\begin{frame}
  \frametitle{WAAPI (Cont.)}

  \begin{itemize}
    \item 减少 CSS 的 parsing 时间
    \item 方便浏览器优化
    \pause
    \item 可以并行生成多个不同属性的 Transition
    \pause
    \item \textbf{可以和 Houdini 一起用!}
    \pause
    \begin{itemize}
      \item 可以并行生成多个相同属性的 Transition
    \end{itemize}
  \end{itemize}
  
  \pause
  \vspace{2em}

  非常适合用来写:FLIP,进入/离开过渡
\end{frame}

\begin{frame}
  \frametitle{ResizeObserver \& IntersectionObserver}
  \includegraphics[width=\textwidth]{assets/OBSERVE.jpg}
\end{frame}

\begin{frame}
  \frametitle{ResizeObserver \& IntersectionObserver}

  检测元素的尺寸、相对于一个父元素的位移变化
  
  \vspace{2em}

  \begin{itemize}
    \pause
    \item 和 Canvas 配合使用
    \pause
    \item 和 \texttt{position: sticky} 配合使用
    \item ...
  \end{itemize}

  \vspace{2em}
  \pause

  相比于 \texttt{resize} 和 \texttt{scroll} 事件,减少 polling。

  \vspace{2em}
  \pause

  BTW: \texttt{MutationObserver}
\end{frame}

\begin{frame}
  \frametitle{Also worth reading:}
  \begin{itemize}
    \item CSSOM
    \begin{itemize}
      \item 可以提供 Houdini 的一个 Fallback
      \item 可以在不强制 Repaint 的情况下从 CSS 得到部分属性
      \item CSS Paint API\footnote{\url{https://developer.mozilla.org/en-US/docs/Web/API/CSS_Painting_API/Guide}}
    \end{itemize}
    \item Grid Layout
  \end{itemize}
\end{frame}

\section{喵喵's practice}
\begin{frame}
  \frametitle{喵喵's practice}

  喵喵喜欢的写 CSS 的方式:
  \begin{itemize}
    \item BEM 命名方式(喵喵裁剪版)
    \item SCSS
    \item 使用 \texttt{@import}
  \end{itemize}
\end{frame}
\begin{frame}[fragile]
  \frametitle{BEM}
  \begin{block}{Block, Element, Modifier}
    \begin{minted}{css}
      .button--primary__text--hint { }
      /* .button.primary .text.hint */
    \end{minted}
  \end{block}

  \vspace{2em}
  \pause
  \includegraphics[width=\textwidth]{assets/frame-full.jpg}

  减少检查 Selector 的时间:不需要检查父子、兄弟关系。
\end{frame}
\begin{frame}[fragile]
  \frametitle{BEM (Cont.)}
  例外:需要使用 Pseudo-element 的时候。

  \begin{minted}{css}
  .container:hover .title { opacity: 1; }
  .paragraph::first-leter { font-weight: 900; }
  .timestamp:before { content: "TIMESTAMP >"; }
  \end{minted}

  \vspace{2em}
  \pause

  减少 JS 的使用,“样式就写在样式表里,和 JS 无关”

  而且还快。
\end{frame}

\begin{frame}[fragile]
  \frametitle{SCSS}

  \begin{itemize}
    \item 和 BEM 相性很好
    \item 有好多内置的函数(生成色板,etc.)
    \item @for
  \end{itemize}

  \vspace{1em}
  \pause

  \begin{minted}{scss}
    @mixin with-title($active) {
      &__title {
        font-size: 1.4em;
        &--#{$active} {
          font-size: 1.8em;
        }
      }
    }
    .post { @include with-title("foo"); }
    .list { @include with-title("bar"); }
  \end{minted}
\end{frame}

\begin{frame}
  \frametitle{Alternatives}

  \begin{itemize}
    \item CSS-in-JS
    \item styled-components
  \end{itemize}

  \vspace{2em}
  \pause

  喵喵不用的原因:
  
  样式变化的时候可能需要 parse 注入的 CSS,编辑器支持比较差,浏览器难以优化,难以调试,etc.
\end{frame}

\begin{frame}
  \frametitle{That's All!}

  \begin{center}
    \includegraphics[width=.5\textwidth]{assets/look.png}

    Question time!
  \end{center}

  \url{https://meow.c-3.moe/sth-about-jielabs}

  \url{https://meow.c-3.moe/writing-meow-plus}
\end{frame}
\end{document}
